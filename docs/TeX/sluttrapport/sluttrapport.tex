\documentclass[nynorsk,12pt,a4paper,oneside]{book}
\usepackage[utf8]{inputenc}
\usepackage[T1]{fontenc}
\usepackage{lmodern}
\usepackage{graphicx}
\usepackage{babel}
\usepackage{hyperref}
\usepackage{subfiles}
\usepackage{pdflscape}
\usepackage[final]{pdfpages}
\usepackage{IEEEtrantools}
%\usepackage{biblatex}
%\usepackage{makeidx}

%\usepackage[toc]{glossaries}

\title{Sluttrapport \\ Dynamisk Nettverksbrannmur}
%\subtitle{Bacheloroppgåve 20E}
\author{Espen Gjærde \and Svein Ove Undal}
%\institute{Høgskolen i Sør-Trøndelag \\ Avdeling for Informatikk og e-Læring}
\date{2013}

\begin{document}
\bstctlcite{trans}
\includepdf[pages=1]{apx/20E.pdf}
\frontmatter
\maketitle
\chapter{Forord}
\paragraph*{}
Denne rapporten er ein del av bacheloroppgåva til forfattarane, og markerer avsluttinga av tre år på Dataingeniørlinja hjå Avdeling for Informatikk og e-Læring ved Høgskolen i Sør-Trøndelag. 
\paragraph*{}
Dette prosjektet er ei vidareføring av Svein Ove Undal sitt prosjekt <<prosjektnamn>> i faget <<RMI med prosjekt i distibuerte systemer>>. Svein Ove Undal ønska då å vidareføre denne oppgåva som bachelorprosjekt, og det var no Espen Gjærde kom inn i bildet. Begge forfattarane går på studieretninga <<Nettverksakitektur og -design>>, og er svært interessert i både nettverksteknologi og Linux. Det var derfor eit enkelt val når vi kunne bruke siste semester på å grave oss ned i desse fagfelta. 

Sidan vi bestemte oss for å nytte Python som programmeringspråk i løysinga av oppgåva, og sidan ingen av forfattarane hadde noko erfaring med dette frå før, måtte vi bruke meir tid enn venta på å eiga opplæring. Sjølv om det i starten kunne sjå litt omfattande ut, henta vi inn mykje tid mot slutten når vi såg kor enkelt det var med eit programmeringspråk over heile fjøla, i staden for å blande bash og php. 

\paragraph*{}
Før vi tek til på sluttrapporten vil vi nytte høvet til å takke rettleiar for bacheloroppgåva, Helge Hafting, for gode råd og innspel undervegs. Vi vi også rette ein stor takk til TIHLDE-LAN og spesielt deltakarena for tolmod og gode tilbakemeldingar under testinga av systemet. 
 
\begin{figure}[b]
Trondheim, \today \\
\vspace{1cm}
\end{figure}
\begin{table}[b!]
	\centering
	\begin{tabular}[c]{ c p{4cm} c } 
		%\emph{(sign.)} & ~ & \emph{(sign.)} \\ 
		\cline{1-1} \cline{3-3}
		\parbox{5cm}{\centering \textsc{Espen Gjærde}}
		& & 
		\parbox{5cm}{\centering \textsc{Svein Ove Undal}} \\
	\end{tabular}
\end{table}

\part{Oppgåva}

\chapter{Oppgåveteksten}

\section{Orginal oppgåvetekst}
\emph{<<Oppgava går ut på å lage ein streng gateway/brannmur der kun autentiserte brukarar kan logge seg på. Brannmuren skal dekke kjente problem med likandes system som mac-adresse spoofing og dns-tunnelering.>>}
\paragraph*{}
Denne oppgåveteksten vart revidert for å utvide oppgåva til to personar. Aktuelle utvidingar var då automatisk justering av bandbredde til kvar enkelt brukar, administrasjonspanel og støtte for IPv6.

\section{Revidert oppgåvetekst}
Ein ny og revidert oppgåvetekst vart laga, og går som følgjer:
\paragraph*{}
\emph{<<Lage ein gateway med brannmur som tvingar brukarar til å logge seg på før dei har tilgang til internett. Gatewayen skal automatisk justere bandbredde til den enkelte brukaren. Administratorar skal ha tilgang til eit oversiktleg administrasjonsgrensesnitt.>>}


\chapter{Samandrag}
\paragraph{Treige nettverk} er noko nesten alle har vore bort i, og irritasjonen over ei treig nettilkopling er noko mange kjenner til. Dette rammar oss ofte når vi nyttar oss av ymse offentlege -- gjerne trådlause -- nettverk. For oss studentar som gjerne bur i kollektiv og dermed deler nettlinja med fire-fem andre brukarar har også ofte kjent på korleis det er når nokon <<stjel>> nettlinja. Det er dette problemet vi vil til livs med denne bacheloroppgåva.
\paragraph{}Ved å tvinge brukarane til å først logge seg inn, kan vi følje med på kor mykje av linja kvar enkelt legg beslag på, og ta affære om nokon øydelegg for andre. Slik kan vi sikre at alle får ein betre nettopplevelse og nettlinja vert rettferdig delt mellom brukarane. Det som skil dette systemet frå andre, er at vi ikkje sett avgrensar nettlinja til nokon av brukarane før bruken vert eit problem. Er du aleine på nettverket skal du sjølvsagt få heile linja -- og brukar dei andre brukarane berre ein liten del av linja, er det ikkje noko problem at du tek resten av linja. Avgreninga skjer først når bruken er større enn kapasiteten. Da må vi sjå til at alle får lik utnytting av linja. 
\paragraph{}Det er også fleire ressursar enn berre bandbredde i ei nettlinje. Antall oppkoplingar kan fort bli eit problem for nokon rutarar. Derfor har vi tre <<dimensjonar>> vi ser på når vi avgrensar bruken. Dei to første bandbreidda opp og ned, men vi ser også på antall oppkoplingar. Ut i frå erfaringar vi fekk på <<TIHLDE-LAN>> gir antall oppkoplingar også ein god indikasjon på kven som nyttar mest av linja. Om nokon til dømes brukar torrent-teknologi, vil dei gjerne utmerke seg i statistikken med svært mange oppkoplingar. 

\paragraph{Utviklingen} av systemet er gjort etter rammeverk og utviklingsprosessar som vi er kjent med frå faget <<Systemering med prosjekt>>, og vi har valt utviklingsprosessen UP. Vi fekk også testa delar av systemet under THILDE-LAN. Dette gav oss mange gode erfaringar, og vi fekk samla mykje trafikkdata som gjorde det enklare for oss å kjenne kvar skoen trykkjer. Vi fekk også prøve oss på XP-utvikling, da det sjølvsagt oppstod nokre bugs. Da var det berre å kaste seg rundt å komme med ei løysing. 

\tableofcontents
\mainmatter
\part{Sluttrapporten}
\chapter{Introduksjon}
\section{Introduksjon}

\section{Definisjonar og forkortingar}
\subsection{Ordliste}


%\printglossaries

\paragraph{Apache} eller Apache HTTP Server er ein populær http-tjenar. Denne sørgjer for at nettsidene til systemet verte vist fram.
\paragraph{Django} er eit rammeverk som gjer det enkelt å bygge dynamiske nettsider med python.
\paragraph{Python} eit populært høgnivå kodespråk. Det er dette det meste av kodinga for prosjektet er gjort
\paragraph{Bash(-script)} Bourne-Again-SHell. Er ein komandotolkar for unix-system. Det kan også setjast opp ei fil med komandoar som skal utførast av bash. Dette er eit bash-script.
\paragraph{NAT} står for <<Network Address Translation>> og er ei fellesbenevning på oversetting av ip-addresser mellom ulike nettverk. 
\paragraph{Port-Forward} er ei form for NAT der vi oversett portnummer. Med Port-Forward kan vi rute om trafikk frå ein spesiell TCP- eller UDP-port til ein annan.
\paragraph{IPv4} er ei forkorting for <<Internet Protocol version 4>> og er i dag den IP-versjonen som er mest utbredt. 
\paragraph{IPv6} er ein ny versjon av IP, med mange fleire tilgjenelege addresser enn IPv4. 
\paragraph{Distro} er ei forkorting/slang-form av distribusjon og blir brukt om ei avgreining eller ein type linux.
\paragraph{WebUI} Web User Interface -- eit nettbasert brukargrensesnitt. 
\paragraph{DHCP} \emph{Dynamic Host Configuration Protocol} er ein protokoll for utdeling av nettverkskonfigurasjon i eit nettverk. 
\paragraph{Linuxkjerna} 
\paragraph{CLI} \emph{Command Line Interface} eller <<ledetekst>> på norsk er eig grensesnitt for å sende kommandoar direkte til eit program. Det vert også ofte brukt om brukergrensesnitt som er basert i ein kommandoterminal.




\chapter{Teori}
I denne delen av sluttrapporten vil vi gå gjennom teknologien vi har tilgjeneleg for å løyse oppgåva, og gje ei oversikt over dei konsepta vi kan bruke. 

\section{Konsept}
\subsection{Captive-Portal}
\emph{Captive-Portal} er eit samleomgrep for nettverksløysingar der brukarar vert tvinga til å gå gjennom ei påloggingsprosedyre i nettlesaren. Det er ikkje nokon fast definisjon på ein Captive Portal. I følgje Cisco skal ein captive-portal videresende ein nettlesar til ein bestem plass, ta vare på orginal HTTP-informasjon og \emph{kan} videresende brukaren basert på denne informasjonen. \cite{ciscoCP}

\subsection{Behandling av trafikk}

\subsection{Tidsavbrot for sesjon}
Når nokon har logga seg på eit system, er det ikkje sagt at vedkommande hugsar å logge seg ut. I eit høve der vi ikkje vil at uvedkommande skal kunne bruke eit system, eller der vi vil vere sikker på at vi har knytt rett brukarnamn til kvar sesjon, er det eit poeng at brukarar som ikkje er aktive vert logga ut. 

\section{Aktuell teknologi}
Når vi no ser på dei forskjellige teknologiane vi kan nytte, er det naturleg å dele desse inn i operativsystem, kodespråk og programvare. Vi vel også å skilje ut viktige modular som er inkludert i Linuxkjerna og forklare desse nærmare.
\subsection{Operativsystem}
\subsubsection{Debian 6.0.6 <<Squeeze>>} 
Vi har valt å nytte oss av linux-systemet Debian. Vi nyttar versjon 6.0.6, som er den versjonen som var den siste stabile utgivinga av Debian då vi starta med bacheloroppgåva. Denne har kodenavnet <<Squeeze>>. Det har i ettertid (3.mai) kome ut ein ny stable-versjon av Debian, <<Debian 7 Wheezy>>. \cite{wheezy}

\subsection{Programmeringsspråk}
\subsubsection{Python}
Kodespråket Python er eit høgnivå kodespråk som kjem ferdiginstallert i dei fleste linux-distibusjonar. I Linuxversjonen vi nyttar er Python versjon 2.6.6 integrert. \cite{pythonversjon}. 
Python har gode moglegheiter for å samarbeide med linux-systemet, og er eit kodespråk som <<plays well with others>>, altså er enkelt å setje opp mot andre systemer, slik som database og nettjenarar. \cite{python, pythonapps} I tillegg til dette er det også enkelt å gjere systemkall i linux frå Python. \cite{pythonsubprocess}


\subsubsection{Django}
Django er eit rammeverk for utvikling av nettsider i python. Å nytte dette rammeverket gjer at vi enkelt kan lage gode dynamiske nettsider. Django har gode modular for oppsett av nettsider, noko som gjer at vi slepp å lage mange nokså like modular. I staden kan vi bruke velkjente og gjennomtesta metodar og heller tilpasse desse til våra bruk. Dette sparar oss for mykje koding, blant anna av sesjonshandtering.\cite{djangowww, djangosessions}

I dette prosjektet nyttar vi Django versjon 1.2.3, da det er denne som følgjer pakkebrønnen\footnote{Pakkebrønn: ein katalog av testa og godkjente programvare for eit system. \\For Debian:  \url{http://packages.debian.org} } til Debian Squeeze. \cite{djangopkg}

\subsubsection{Bash-scripting}
Bourne-Again SHell er eit kommandoskall, og er det komandoskallet som blir nytta som standard i debian linux. \cite{debbash, bashwww} Bash-scripting er ei liste med kommandoar som skal utførast av bash. Bash støttar også scripting språka til mange andre kjente og nytta unix-kommandoskall, deriblandt csh og ksh. \cite{bashabout}.

\subsubsection{PHP}
Hypertext Preprocessor er eit kodespråk for å lage dynamiske nettsider. Dette er eit poplært kodespråk som er godt utbredt i open kjeldekode-miljø. PHP er laga for at utviklarar hurtig skal kunne lage dynamiske nettsider. \cite{phpwww} Versjonen som er aktuell for Debian Squeeze er 5.3.3 \cite{phppkg}

\subsection{Programvare}
\subsubsection{Apache HTTP Server}
Apache HTTP Server er den mest brukte http-tenaren på internett. \cite{apahcetraf}. Dette er ein robust vevstjenar som er basert på open kjeldekode. Dette har gjort at der er mange modular tilgjeneleg. Blandt desse har vi mod\textunderscore wsgi for Python og php5\textunderscore module for PHP. \cite{modwsgi, phpmodule}

\subsubsection{MySQL Community Edition}
MySQL er verdas mest populære opne relasjonsdatabase, og er spesielt vanlig å nytte til webapplikasjonar. Relasjonsdatabasen gir oss muligheten til å hurtig og effektivt lagre og hente data. \cite{mysqlwww} 

\subsubsection{SQLite}
SQLite er ein filbasert relasjonsdatabase som er vanleg å nytte i frittståande applikasjonar. Den er vanleg å bruke til enkle nettsider og små applikasjonar. SQLite samlar all databaseinformasjonen i ei fil, og gjer det soleis enkelt når det kun er ein applikasjon som skal nytte databasen\cite{sqlabout}. Bakdelane med denne programvara er at der er ein del begrensingar i SQL-funksjonalitet, heriblant JOIN og ALTER. \cite{sqlimit}

\subsubsection{Internet Systems Consortium DHCP Server}
Internet Systems Consortium (ISC) DHCP Server er ein implementasjon av DHCP-systemet og står for utlevering av nettverksinformasjon til datamaskiner i eit nettverk. Noko av det mest vanlege er IP-addresser, DNS-informasjon og kva som er addressa til nærmaste router. Denne serveren har også moglegheit for å kjøre kommandoar når spesielle hendinga skjer \cite{dhcpconf, dhcpdman}. 

\subsection{Integrerte linuxmodular}
\subsubsection{Netfilter og IPtables}
Netfilter er et rammeverk for filtrering av nettverkstrafikk, og har vore ein del av linux-kjerna sidan linuxversjon 2.4 \cite{netfiltercore}. Gjennom netfilter og kommandoen <<iptabels>> gjer dette linux i stand til å mellom anna utføre NAT, pakkefiltrering for både IPv4 og IPv6, føre statistikk over oppkoplingar og utføre port forwarding. 

\subsubsection{Linux Advanced Routing and Traffic Control}
Linux Advanded Routing and Traffic Control (lartc) er eit sett med verktøy for å manipulere pakkar som går gjennom linux, og er ein del av kjerna i linux si pakkehandtering. \cite{lartcintro} Denne modulen gir mellom anna moglegheit for å merke pakkar slik at dei kan behandlast i andre verktøy seinare. 

\subsection{Protokollar}
\subsubsection{DHCP}
DHCP er ein protokoll for utsending av nettverkskonfigurasjon til klientar i eit nettverk. Denne protokollen fungerar ved at ein klient sender ut ein førespurnad om DHCP-informasjon og ein server svarer på desse. Klienten vil deretter automatisk få IP-addresse og vanligvis også få tilsendt innstillingar for standardruter i nettverket og informasjon om navnetjenarar.  \cite{datakom} 
For å unngå at det <<hopar seg opp>> med inaktive klientar, har desse automatiske innstillingane eit tidsavbrot. Det er klienten sitt ansvar å seie frå om at den framleis er aktiv. Dersom dette ikkje vert gjort, kan tenaren sende ut desse innstillingane til nye klientar. \cite{rfc2131}

\section{Utviklingsprosess}
\subsection{Arbeidsmetodikk}
Her vil vi raskt gå gjennom nokon av dei arbeidsmetodikkan som var aktuelle å bruke i prosjektet. Alle desse er smidige og iterative prosessar. 

\paragraph{Scrum} er ein arbeidsmetodikk som passar best for større multidisiplinære team. Prosessen involverer kunden etter hver iterasjon, eller <<sprint>> som det heter i Scrum-terminologien. Hver sprint varer typisk i to eller tre veker og ender med en fremvising av et ferdig produkt. \cite{scrumprimer}

\paragraph{eXtreme Programming} (XP) er en utviklinsmetode som vil minimere tid brukt på design og planleggingsfasen, og heller fokusere på ofte og gode tilbakemeldingar frå kunde. Eit poeng er også å \emph{<<do the simplest thing that could possibly work>>}, \cite{xpbook}  og at ein skal programmere i lag, for å unngå misforståingar. 

\paragraph{Unified Process} (UP) består av fire fasar. \emph{Oppstart} som er utviklinga av visjon og grove planskisser for prosjektet, \emph{utgreiing} av krav og utvikling av kjernefunksjonar, \emph{konstruksjon} av resterande funksjonalitet og start av implementasjon, og \emph{overgangen} til betatesting og utrullinga av programvaren. \cite{upxpuml} Det må understrekast at desse fire fasane ikkje er fire iterasjonar, men at kvar av desse fasane kan inneholde fleire iterasjonar. Det er også slik at vi ikkje er sekvensielt ferdig med prosessane (les: fossefall). Til dømes er ikkje visjonsdokumentet ferdig etter oppstartsfasen, men vert revidert gjennom heile prosessen. 

\subsection{Kodefilisofi}
I unix-filisofien heiter det seg at program skal <<gjere éin ting, og gjere den vel>>. Ein skal altså ha små men velfungerande modular, som skal fungere uavhengig av kvarandre \cite{unixprog}. På denne måten kan ein sikre at endringar i ein modul ikkje slår ut heile systemet, samstundes som at ein lagar eit fleksibelt og lett "hackbart"\footnote{Hack som i å nytte systemet på alternativ måte, ikkje hacking som i datainnbrot.} system. 

\paragraph{Frå objekt-orinentert} programmering har vi også M-V-C prinsippet, som i grove trekk går ut på å dele koden i tre hovuddelar: \cite{mvc}

\begin{description}
	\item[Model] Som tek seg av kommunikasjon med ytre system som database, filsystemet, kjernefunksjonar osv.
	\item[View] er framvisinga av informasjon og innhenting av data frå brukaren
	\item[Controller] gjer all kontroll og konvertering av data mellom framvisinga og det bakanforliggande systemet.
\end{description}

\paragraph{}
Ved å helde oss til desse prinsippa kan vi både lett halde oversikta når systemet verte komplekst, og enklare dele ansvarsområde for programmeringa mellom utviklarane.

\chapter{Metode}
I dette kapittelet vil vi gå gjennom og grunngi dei forskjellige vala av teknologi og greie ut om arbeidsprosessen fram mot det ferdige produktet. 
 
\section{Val av teknologi}
\subsection{Operativsystem}
\paragraph{}
Valet av operativsystem var eigentleg meir eit val mellom linuxdistribusjonar. I Windowsverda får vi ikkje samme kontroll over kva modular vi vil ha med, og kan dermed ikkje på same måte skreddersy operativsystemet til det behovet vi har. I tillegg er dei fleste linuxdistibudjonar fritt tilgjeneleg, i motseting til Microsoft-system. 
\paragraph{}
Debian er eit stabilt og anerkjent linuxsystem, og er <<mor>> til mange andre distroar, deriblant den brukarretta versjonen Ubuntu. Når vi vel å kjøre debian ar det to hovudgrunnar til dette. Ein av dei er at Debian har eit stort og svært aktivt utviklermiljø som prioriterer tryggleik og stabilitet. Dette gjer at vi kan stole på programvaren som følg med, men også at vi ikkje får siste skrik når det gjeld versjonar. Til dømes nyttar vi Python 2.6.6 sjølv om siste versjon av Python2 er 2.7.4, og Django 1.2.3 sjølv om siste versjon er 1.5. 
Den andre hovudgrunnen til at vi vel Debian er så enkel som at begge utviklarane har mykje erfaring med systemet. Det er også dette systemet som vert anbefalt for kurset <<Linux Systemdrift>> ved Høgskolen i Sør-Trøndelag.
\paragraph{}
Når dette er sagt, er der ikkje nokon grunn til at systemet vi har utvikla ikkje skal fungere på andre distribusjonar. Dei linuxmodulane vi nyttar er felles for dei fleste system, og kommandoane er like.  

\subsection{Programmeringsspråk}
Valet her har vore mellom å lage eit sett av bash-script med PHP som WebUI, og å kode eit system i Python. Vi har valt å nytte oss av høgnivå-språket Python med rammeverket Django for WebUI. Dette gjer at vi kan kode alt i eit programmeringspråk og gir ein enklare og tryggare måte å kommunisere mellom dei forskellige modulane. Å nytte Python gjer det også naturleg å skrive objektorientert kode og dele inn systemet etter MVC-prinsippet. 
\paragraph{}
Det systemet som var utvikla gjennom Svein Ove Undal sitt haustprosjekt var laga med bash-script og PHP. Det hadde derfor vore eit enkelt val å fortsette med dette, og berre bygge på meir på dette systemet. Ettersom vi også såg etter måtar å utvide oppgåva på vart det naturleg å legge systemet om til Python. Dette gav også utfordringar i form av opplæring.
\paragraph{}
Ingen av utviklarane hadde frå før mykje erfaring med Python, noko som førte til at vi måtte bruke litt tid på å lære oss programmeringsspråket. Vi brukte noko tid på dette før vi endeleg bestemte oss for Python. Når vi etterkvart såg kor veleigna og enkelt dette språket var, bestemte vi oss også for å sjå nærmare på å nytte Django i staden for PHP. Det viste seg fort at å nytte Django gav mange gode effektar. Sidan Django har ein del ferdigutvikla modular for sesjonshandtering på nett, kunne vi konsentrere oss meir om kjernefunksjonane i systemet.

\subsection{Programvare}
\paragraph{}
For å få betre kontroll på brukarane av systemet og for å kunne bruke DHCP-hendingar som utløysar for aksjonar i koden, har vi valt å nytte ISC DHCP-server. Dette var også eit naturleg val da dette er den mest utbredete dhcp-serveren i *nix-verda. ISC DHCP-server sin funksjon for å utnytte DHCP-protokollen si mekanisme for fornying av TCP/IP-innstillingar gjer denne programvara spesielt egna. I tillegg er dette ei programvare som begge utviklarane har erfaring med frå kurset <<Linux Systemdrift>> ved Høgskolen i Sør-Trøndelag. 
\paragraph{}
For å effektivt kunne lagre og hente ut data og statistikk frå systemet nyttar vi relasjonsdatabasen MySQL. Her var det også fleire andre alternativ, bl.a. har Django innebygd støtte for SQLite. Å bruke SQLite ville også gjort at vi ikkje var avhengig av ein ekstern databaseserver, da SQLite lagrar alt i ei lokal fil. Når vi likevel valde å nytte MySQL var dette i hovudsak fordi vi har erfaring med denne frå før, og soleis var sikker på at alle databasefunksjonar vi hadde tenkt oss ville fungere.  
\paragraph{}
Når vi også har valt å nytte Apache HTTP-server er det også tidlegare erfaingar som gjer at vi vel denne. Begge utviklarane har erfaring med konfigurasjon og oppsett, og har også her lært om denne i faget <<Linux Systemdrift>>. 

\paragraph{}

\section{Systemarkitektur og strukturering av kode}
Her vil vi gå gjennom de valgene som er gjort når det gjelder kodefilosofier og måter å bruke de forskjellige modeulene og programvarene på.

\subsection{Kodestruktur}

\subsection{Tekniske løysingar}
\paragraph{Tvungen pålogging} til systemet blir gjort ved å lukke brannmuren for alle andre enn brukarar som alt er pålogga. Deretter brukar vi NAT for å avskjære trafikken og sende brukarane til ei påloggingside. Etter at brukarane her har logga på, åpnar vi brannmuren og fjernar avskjæringsfunkjonen for brukaren. Systemet vil no gå over i ein vanleg brannmur, gateway og ruter-funksjon for brukaren. 

\paragraph{} For å unngå at brukarar skal jukse seg gjennom brannmuren ved å stele andre sine IP-addresser krev vi at alle adresser er utdelt av DHCP-serveren. Ved innlogging vert adressa og mac-adressa registert og sjekka mot DHCP-serven sitt register. Det er kombinasjonen av brukar, passord og IP-addresse som vert sett på som ein unik brukar, men ved å stille inn konfigurasjonsfila for systemet kan administrator tillate fleire innloggingar pr brukar.

\paragraph{Automatisk utlogging ved tidsavbrot} gjer vi ved å utnytte oppførselen til klientar av DHCP-protokollen. I DHCP-serveren  si konfigurasjonsfil legg vil til nokre linje som gjer at CLI-kommandoen for utlogging av brukarar blir kjørt for IP-addresser som ikkje fornyar DHCP-leasen sin. På denne måten sikrar vi at pcar som ikkje er kopla til nettet automatisk vert logga ut. 

\paragraph{Overvåking} av linja gjer vi fortløpande, og for å spare på ressursar overvakar vi i utgangspunktet kun totaltrafikken, men loggar trafikkmengda for kvar brukar. Først når systemet oppfattar at linja ut har kapasitetsproblem ser vi på trafikkmengda for kvar brukar, og avgjer om nokon må avgrensast. Vi vil etter dette følge brukaren, og sleppe brukaren fri frå avgrensingane når trafikkmengda frå brukaren er på eit akseptabelt nivå. 

\paragraph{Avgrensinga} vert gjort ved at vi i brannmuren har laga eit sett med reglar som fungerar som eit <<fengsel>>. Brukaren er ikkje lenger fri til å bruke kapasitet, og all trafikk frå brukaren vert først flagga når dei kjem inn til systemet, og trafikk som går ut over fastsette grenser blir forkasta. .. ICMP-TIMEOUT? eller berre dROP?


\section{Prosessen}
\paragraph{}
Som utviklingsprosess har vi valt å nytte \emph{Unified Process} (UP). Som alternativ til denne var også \emph{SCRUM} nevnt, men utviklarane såg ikkje på dette som ein føremålstenleg utviklingsprossess når det var tre personar -- to utviklarara og ein rettleiar -- involvert. Vi var også inne på tanken om å nytte \emph{eXtreme Programming}, men sidan begge utviklarane har mest erfaring med UP, falt valet på denne utviklingsprosessen.
\paragraph{}
Vi laga ikkje noko tradisjonelt GANT-diagram for utviklingsprosessen, men laga i staden ein tabell med mål og tidsfristar for når forskjellige utviklingsteg skulle vere ferdig. Vi jobba deretter iterativt fram mot desse, og har testa kvar modul undervegs. Dette vart gjort for vi av erfaring veit at det kan bli lite tid til testing mot slutten, og da var dette ein måte å være sikker på at testing vart gjennomført for kvar modul. 
\paragraph{}
Det ligg ikkje føre noko forstudierapport for dette bachelorprosjektet, da det er ei viderføring av eit haustprosjekt, og forstuderapport dermed ikkje var påkrevd. 
\paragraph{}
Første del av prosessen gjekk for det meste med til eiga opplæring, både i Python og deretter Django-rammeverket. Deretter brukte vi tida på å oversetje bash-script frå haustprosjektet til fungerande Python-program. Når dette var utført fekk systemet på mange måtar ein ilddåp, sidan vi skulle teste dette på TIHLDE-LAN . Dette var tidleg i prosessen og mykje av programvara var uferdig. Vi hadde CaptivePortal-funksjonaliteten på plass, og konsentrerte oss dermed om å lage statistikk-funksjonalitet for å samle mest mogleg data.
\paragraph{}
For å få testa mest mogleg funksjonalitet -- og rette nokon feil -- nytta vi element frå utviklingsprosessen eXtreme Programming under TIHLDE-LAN. Under lanet vart mange av metodane for avgrensing av bruk og samling av trafikkdata utvikla og testa. Vi brukte her ei form for parprogrammering der ein programmerte, medan den andre las over koden før den deretter vart testa. Dette gav gode resultat om ein ser på antall utvikla modular og linjer, men gav også eit stort etterarbeid i for dokumentasjonsdelen.
\paragraph{}


\chapter{Resultat}

\section{Faglige resultat}
\subsection{Måloppnåing}
	
\subsection{Kommentar til UseCase-ar}
(UC)
\subsubsection{UseCase 1: Logg inn}
Innlogginga er fungerande og enkel for brukaren å utføre. \\
scpr. innlogging?

\subsubsection{Use Case 2: Vis statistikk}
Når kvar brukar har logga seg inn, vil nettoppslag mot innloggingstjenaren videresende brukaren til ei statistikkside. Her vi brukaren kunne sjå kor mykje han har lasta ned totalt, kor mange tilkoplingar han brukar og kva som er gjennomsnittleg linjebruk for økta. Brukaren vil også få informasjon her dersom det er sett restriksjonar for linja han har.

\subsubsection{Use Case 3: Administrer Brannmur}
Administrator vil få tilgang til eigne administrasjonsider der dei kan sjå på innstillingane, få oversikt over aktive brukarar og eventuelt legge til nye. Her er også ei eiga side for å legge til eller fjerne reglar i brannmuren. Noko som ikkje kjem fram i UCen, men vart naudsynt er at brukaren først må velje kva del av brannmuren han vil lage regel for. Vi har eit sett reglar for innkommande til server, og eit sett for trafikk som skal vidaresendast.

\section{Administrative resultat}

\subsection{Krav og retningslinjer}
Det var i visjonsdokumentet sett opp ei rekkje krav til utforming av både rapportar, kode og bruk av språk. Vi vil ikkje nemne alle her da visjonsdokumetet ligg vedlagt, men dreg fram følgande:

\subsubsection{Programmeringsstandarder}
Programmering er gjort i Python og HTML, samt noko SQL der det var naudsynt med databasespørringar. For HTML følge vi Standard for HTML 4.01 Strict, og Python føljer PEP8-standarden for pythonkoding. 

\chapter{Drøfting}
\section{Kvifor dette resultatet?}
\section{Svakheiter ved produktet}
\section{Styrkar ved produktet}
\section{Vidareutvikling av produktet}
\chapter{Konklusjon}
\chapter{Kjelder og referansar}
Mesteparten av litteraturen er på engelsk. Vi har derfor valt å merke det som er på norsk med [NORSK]. Dersom denne merkinga ikkje er i referansen, er kjelda engelsk. 

\bibliographystyle{IEEEtran}
\bibliography{IEEEabrv,kilder}{}

\part{Vedlegg}
\appendix 
\chapter{Testrapport frå TIHLDE-LAN}
\chapter{Visjonsdokument}
	Nummereringa på sidetala følgjer no vedlagt dokument.
	\includepdf[pages=1-19]{apx/visjonsdokument.pdf}
	\includepdf[pages=1,landscape]{apx/rosanalyse.pdf}
\chapter{Kravdokument}
	Nummereringa på sidetala følgjer no vedlagt dokument.
	\includepdf[pages=-]{apx/kravdokument.pdf}
\chapter{Arkitekturdokument}
	Nummereringa på sidetala følgjer no vedlagt dokument.
  	\includepdf[pages=-]{apx/arkitekturdokument.pdf}
\chapter{Prosjekthandbok}
	Nummereringa på sidetala følgjer no vedlagt dokument.
%	\includepdf{apx/prosjekthandbok.pdf}
	\section*{HER KJEM ET DOKUMENT!}
	Nummereringa på sidetala følgjer no hovuddokumentet
\chapter{Dokumentasjon frå haustprosjektet}
		Nummereringa på sidetala følgjer no vedlagt dokument.
	\includepdf[pages=-]{apx/sluttrapport_sou.pdf}
	Nummereringa på sidetala følgjer no hovuddokumentet

\backmatter
\chapter{Etterord}
\section{Takk!}
\paragraph{Helge Hafting} for god rettleiing og gode fagkunnskapar.
\paragraph{TIHLDE} og spesielt TIHLDE-LAN for at vi fekk nytte dei som testkaninar lenge før systemet var produksjonsklart. Dette gav oss mykje gode erfaringar.
\paragraph{Deltakarane på TIHLDE-LAN} som var hyggelege og ga oss verdifulle tilbakemeldingar.

\end{document}
