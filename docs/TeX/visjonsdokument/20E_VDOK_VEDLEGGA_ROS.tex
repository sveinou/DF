\documentclass[nynorsk,12pt,landscape,a4paper]{article}
%\documentclass[12pt]{article}
\usepackage[utf8]{inputenc}
\usepackage[margin=2cm,landscape,a4paper]{geometry}
\usepackage{graphicx}
\usepackage{babel}
\renewcommand*{\familydefault}{\rmdefault}
\title{Risikoanalyse \\ Dynamisk Nettverksbrannmur
}
\author{Espen Gjærde \and Svein Ove Undal}
\date{11.04.2013}

\begin{document}
%\maketitle
\newpage

\begin{table}[h]
\section*{Risikoanalyse }
\begin{tabular}{l l p{5cm} c c c p{9cm}}
	\textsc{ID} & \textsc{Kategori} & \textsc{Hending} & \textsc{Sannsyn} & \textsc{Konsekvens} & \textsc{Risiko} & \textsc{Tiltak} \\ \hline
	A & Personell & Langvarig sjukdom eller anna fråvær & 2 & 3 & 6 & Deling av all dokumentasjon, informasjon, kode og notat \\
	B & Personell & Samarbeidsvanskar i team & 1 & 4 & 4 & Teambuilding, aktivitetar og god kommunikasjon \\
	C & Avhenigheter & Feil på programvare vi er avhengig av & 1 & 3 & 3 & Så langt som mogleg nytte programvare som er i stable-versjon. \\
	D & Test & Produkt ikkje i kjørbar versjon til avtalttestid & 3 & 3 & 9 & Utvikle i små steg, alltid utføre enkle testar etter omprogrammering \\
	E & Programvare & Programvare oppfører seg ikkje som forutsett & 3 & 2 & 6 & Gjere god research, finne eventuelle alternativ \\
	F & Sluttprodukt & Vanskeleg brukergrensesnitt & 2 & 4 & 8 & Testing og prototyping. Få tilbakemeldingar fra ikkje-teknologar \\
	G & Utvikling & Tap av viktig data & 1 & 4 & 4 & Hyppig opplasting av kode, backup.	\\ \\ \hline \hline
\end{tabular}
\end{table}

\begin{table}[h]
\begin{tabular}{c p{10cm} c c p{10cm}}
	\textsc{Sannsyn}	& \textsc{Forklaring} & ~ & \textsc{Konsekvens} & \textsc{Forklaring} \\ \cline{1-2}  \cline{4-5}
	1	& Lite truleg at hendinga skjer. Sjeldnare enn kvart 5. år. & ~ & 1 & Ubetydleg. Skade kan lett utbetrast. \\
	2	& Hendinga kan inntreffe. Skjer omlag annankvart år. & ~ & 2 & Liten konsekvens, kan påvirke tidsfristar. \\
	3	& Det er truleg at hendinga inntreff. Årleg hending. & ~ & 3 & Store konsekvensar, vil påverke gjennomføring og sluttprodukt \\
	4	& Det er vanleg at hendinga inntreff. Skjer meir enn ein gong i året. & ~ & 4 & Katastrofale konsekvensar. Vil vå betydlige konsekvensar for sluttprodukt. \\ \cline{1-2}  \cline{4-5}

\end{tabular}
\end{table}

\paragraph{Utrekning av risiko}
Risikoen blir utrekna som eit produkt av sannsynet for ei hending og konsekvensen av hendinga. Dette gir eit tal mellom 1 og 16, der 1 er lav risiko og 16 er ei katastrofe som kjem til å skje.

\end{document}